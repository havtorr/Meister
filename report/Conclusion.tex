\chapter{Conclusion}\label{conclusion}

This chapter will summarize the previous sections in light of the research questions proposed in the introduction.

%mangler å konkludere. success? 
%resultat av undersøkelse
%svar på RQ
\section{Research Questions}\label{conclusionRQs}


\begin{theorem}
What is the current state of the various visualization tools that are available?
\end{theorem}

There are currently several different tools that can aid a developer in his understanding of programming, ranging from the regular debuggers found in most IDEs, to code analysis tools and execution tracers.
This paper has focused on the tools that provide information from programs as they are running, and has found that most of these tools focus on a few techniques to provide their users with information.
The techniques in use are mostly based on trace-logging, with an analysis of the log after the program has finished execution.
This analysis is then used to provide detailed behavior on the program, often with an interactive navigation of the data.
Other techniques involve visualization with graphs and diagrams, either based on trace-logs, or on run-time-data captured when a program is suspended at a breakpoint.
Not all of the discovered tools were available for download, and might be considered abandoned, while others are still under active development.


\begin{theorem}
Could any of these be integrated into the current teaching environment at NTNU, consisting of Java and Eclipse?
\end{theorem}

Among the tools the tools that were examined, a variety of platforms were supported, although -- due to the focus of the report -- most of them were designed around supporting programs written in Java, and a majority of these were integrated with Eclipse by design.
As such, they could be integrated with several courses at NTNU, both through use during lecture, or by making them a part of the recommended tools.
This would cover most students participating in a course, excepting those who actively choose to not use them.
The most promising tool, JIVE, chosen for its combination of tracing and diagrams, has already been successfully used to explain design patterns to students in a graduate-level seminar \cite[p. 99]{Gestwicki2005}, indicating that it is suitable as an educational aid.


\begin{theorem}
Is there room for improvement in how these tools are used, and the ease of using them? Can the information they provide be refined, or presented in a more understandable way?
\end{theorem}

\Cref{preDiscuss} concludes that JIVE is the tool that is most suitable for use in the target environment, and, as such, it was selected for further study.
As detailed in \cref{jiveEnhance}, there is a potential for improvements in how JIVE works, and in what kind of usage it encourages.
While the existing features are useful as they are, they were found to be lacking in usability.
The diagrams were found to lacking in their description of certain classes, making identification difficult.
Larger programs were also found to quickly grow the sequence diagram to large sizes, making it harder to get an overview.


Some of these points were the focus of the improvements that were implemented in \cref{jiveImpl}, and their results are summarized when answering the last research question.
Other potential improvements were identified during the evaluation, and are detailed in \cref{conclusionFuture}.


\begin{theorem}
Would the use of such tools and any improvements actually be useful for the students, and help them understand the internal interactions in a program?
\end{theorem}

As the results from the user evaluation show, the use of JIVE, or a similar tool, can certainly be of use for students that are having trouble understanding the interactions within the programs they are creating.
While the exact use case varied, all of the students that participated in the evaluation saw some situation where JIVE would have been useful.
Some referred to situations they experienced when they were having the HCI-course, and to experiences as a student assistant, while others imagined the use for generating accurate documentation for projects.


The implemented changes were all well received, and were considered to be an improvement upon the official version of JIVE.
The participants also expressed desires for further changes, some of which were already identified, but not implemented, and some that could be considered as new ideas.


%framtidig arbeid
\section{Future work}\label{conclusionFuture}
This report has explored some of the currently available tools that are designed to help developers get an insight into the software they are working on, focusing on usage in connection with education.
Through the exploration of JIVE, several potential points of improvement were identified, but not implemented due to prioritizing, and a limited amount of time.
Additional improvements were discovered by the evaluation participants during the evaluation.
These modifications are listed below: %in no particular order?


unimplemented stuffs here\\
further improvements on usability.\\
compression of diagram height\\
looser search terms\\
improving filter, tree-view of classes/packages found in workspace with checkboxes for inclusion/exclusion?\\
If gui is a desired use case: Focus on performance, minimize unnecessary work/avoid work as soon as possible.\\


Apart from the implementation of additional modifications, it would be desirable to start the use of JIVE as part of the lecturing, and encouraging students to try the tool for themselves.
This can be considered to be a natural next step in gaining more data on its use, and to better identify and prioritize further refinements of the tool.


STIKKORD:\\
Test in actual teaching, lectures student assistants, etc.

JFX is essentially not usable with JIVE in current state: too many things going on behind the scene\\


