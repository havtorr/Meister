\section{Task Description}\label{Task Description}%endre navn på kap?
~\\
In order to answer the research questions, the report is divided into appropriate sections.
The prestudy in \autoref{prestudy} will attempt to gain an overview of the different methods used by the tools in question, as well as tools that implement them.
Some tools that does not fit into the teaching environment, due to language- or \gls{ide}-support, are still mentioned because they serve as good examples of the functionality they provide.
After gaining an overview, the most fitting tool will be determined through a comparison of functionality and environment will be examined further, detailing its features and identifying areas that may be improved form a usability standpoint.
~\\

Not all potential improvements can be expected to be implemented, as it depends on how the internal workings of the selected tool.
Some cosmetic changes may be easy to implement, while others may require a rewrite of large parts of the system, which ultimately will be considered to expensive, time-wise, compared to the potential benefits.
~\\

Determining the actual usefulness of the selected tool will be done through testing with students, mainly those that are in their second year at the time of writing, but depending on the access to volunteers, students at other levels may be used as well.
~\\